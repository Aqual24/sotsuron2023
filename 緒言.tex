\section{緒言}\label{緒言}
%背景
近年,アクティブ・ラーニングとして授業活動にゲーミフィケーションといわれるゲームの娯楽性要素や,学習要素を盛り込んだシミュレーション等のゲーム(シリアスゲーム)を導入する動きが活発になってきている.
ゲーミフィケーションは楽しさ,目的意識,達成感の充実といったゲームの主要な要素を取り入れることによって授業への参加意欲や充実感の向上のために活用されている.
シリアスゲームはデジタルゲームの一種で主にコンピュータやタブレットなどを使用し,教育・医療・環境といった社会問題の解決を目的として,英語やプログラミング分野では実際に教育現場で活用されている.

%問題点
一方でデジタルゲームのうち,学習目的でない娯楽要素の多いゲームはゲーム依存症やゲーム脳等のイメージがあり,教育的なメリットや学習機会があることは周知されておらず自宅での学習の妨げになる等の悪い印象が広まってしまっている.

リサーチサービスを提供する会社である株式会社アスマークが2014年に行った「ゲームと子どもに関するアンケート調査」\cite{gameanq}では,ゲームで遊ぶことが子どもの発育・成長にどのような影響を与えると思うかという質問で悪い影響があると思うが半数を超え,中でもゲームが嫌いと答えた人の票数は77.2%という結果だった.
意見として前述のゲーム依存症になると考えやコミュニケーションや運動をしなくなる,ゲームで無駄な時間を過ごすより読書して知識・想像力を蓄えたほうが良いと考えがあった.
またこの問題によって保護者からプレイの制限をされることで,ゲームから得られる学習機会の損失になるという問題点がある.

そこで本研究では,学習を主目的としないデジタル娯楽ゲームの印象の改善とそれらの持つ教育的効果の周知を図るために,様々な娯楽ゲームの持つ教育的なメリットをタグ付けしたWebサイトの開発をし,それにより娯楽ゲームに教育効果や学習機会があることを理解したかを保護者へのアンケート調査を行い評価することを目的とする.