\begin{thebibliography}{99}
\bibitem{gameanq} ASMARQ : ``ゲームと子どもに関するアンケート調査'',\url{https://www.asmarq.co.jp/data/mr201409game/}, 2022/8/19参照
\bibitem{tvgame} 坂元 章 : ``21世紀はテレビゲーミング社会 ―娯楽主導から有効利用ヘ―'',特定非営利活動法人日本シミュレーション&ゲーミング学会, 2000年 10 巻 1 号 pp. 4-13, 2000.
\bibitem{シリアスゲーム意味}Growth Engineering : ``WHAT ARE SERIOUS GAMES?'',2016/3/1,\url{https://www.growthengineering.co.uk/what-are-serious-games/#},2022/12/29参照
\bibitem{シリアスゲームIT}ITmedia : ``東京大学大学院情報学環教授 馬場章氏インタビュー前編 ボクらは「桃鉄」で日本地理を、「信長の野望」や「三国志」で歴史を学んだ'',2005/5/31,\url{https://nlab.itmedia.co.jp/games/articles/0505/31/news003.html},2022/12/29参照
\bibitem{依存症}戸部 秀之,堀田 美枝子,竹内 一夫 : ``児童生徒のインターネット,テレビゲーム依存傾向尺度の構成と,小学生から高校生にかけての依存傾向尺度値の横断的変化'',埼玉大学紀要 教育学部 Vol.59 No.2,pp.181-199,2010.
\bibitem{ゲーム脳の恐怖}森昭雄 : ``ゲーム脳の恐怖'',NHK出版,2002.
\bibitem{ゲーム脳}森昭雄 : ``IT社会と子どもの脳ーゲーム脳,ケータイ脳ー'',日本健康行動科学会 第3回学術大会公開特別公演,pp.87-95,2005.
\bibitem{ゲーム脳メディア}AllAbout : ``TVゲームをし続けるとどうなるのか? ゲーム脳の恐怖,''\url{https://allabout.co.jp/gm/gc/50643/},2002/10/9,2023/1/3参照
\bibitem{反ゲーム脳ITmedia}ITmedia : ``東京大学大学院情報学環教授 馬場章氏インタビュー後編 ゲーム脳,言われているのは日本だけ'',2005/6/1,\url{https://nlab.itmedia.co.jp/games/articles/0506/01/news033.html},2023/1/3参照
\bibitem{反ゲーム脳報告}財団法人イメージ情報科学研究所 : ``ゲームソフトが人間に与える影響に関する調査報告書'',2003.
%\bibitem{ゲーム条例}香川県条例 第24号,``香川県ネット・ゲーム依存症対策条例'',
\end{thebibliography}