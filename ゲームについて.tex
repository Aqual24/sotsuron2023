\section{ゲームについて}
\subsection{教育に活用されるゲームと娯楽ゲーム}
本稿で扱うゲームの種類は家庭用ゲーム機やパソコン,タブレット,スマートフォン等でプレイするデジタルゲームである.
またそのゲームは大まかに教育や学習目的のものと娯楽向けのゲームに分けることができる.

\subsubsection{教育に活用されるゲーム}\label{教育ゲーム}
\ref{緒言}で記述したシリアスゲームは学習や社会問題の解決のための専門的に開発されたゲームであり,海外の学校等の教育現場でギガタブ等のコンピュータを利用し導入されている.
例としてKONAMIが国連世界食料計画(WFP)と協力し発売した「Food Force」というゲームがあり,これは世界の飢餓撲滅のための食糧支援の活動が学べるものである.
プレイヤーはWFPの一員となり飢餓地域に食糧を届けるために物資を確保し輸送,緊急事態にも対処しながら実際に行われている支援についてゲームを通して学び,考えを深めることができる.

シリアスゲームの他にも英語やプログラミングなどの学習のためのゲームや算数・数学の図形をシミュレーションするゲームなどがある.

\subsubsection{娯楽ゲーム}
本研究で主に扱っている娯楽ゲームは\ref{教育ゲーム}で述べたような学習が主目的のゲームとは違い,楽しさや達成感,感動,ストレス解消などを得ることが主目的のゲームで娯楽向けのものを指す.
例としてNintendoの「スーパーマリオブラザーズ」や「スプラトゥーン」,「あつまれどうぶつの森」等が挙げられる.

\subsection{ゲームに関する課題}
インターネットやゲーム機器の技術発達に伴い使用者の低年齢化が進み,日常生活の一部や学習,娯楽に使用されるようになった.その反面,過剰な利用や有害な事物に触れることが増えている.
ゲームに関しては過剰利用によるゲーム依存やゲーム脳の問題があり,ニュース等で取り上げられるなどして問題視されている.

\subsubsection{ゲーム依存・ゲーム脳}
ゲーム依存症は正式に「インターネット・ゲーム依存症」や「ゲーム障害」と言い,インターネットやゲームをする時間が長くなり日常生活に支障を来し,他のことに興味を失ったり考えることが限定的になったりする症状が出る病気である.
これにより家族や友人関係が良好でなくなり健康にも害が生じる例がある.

ゲーム脳は日本大学の教授である森昭雄氏が2002年に出版した「ゲーム脳の恐怖」\cite{ゲーム脳の恐怖}で提示された造語で,コンピュータゲームや携帯電話・パソコンを操作することにより脳が「痴ほう症」に近い状態になるなど悪影響を与えるというものである.
ただ,ゲーム脳といわれているのは日本だけで同じような状況は映画鑑賞などでも起きるという意見\cite{反ゲーム脳ITmedia}や子ども時代の環境等の他の要因が強いのではないか\cite{反ゲーム脳報告}ということから疑問視されている.
またテレビやインターネット等のマスメディアで肯定的に取り上げられたため\cite{ゲーム脳メディア}世間に広まった.

\subsubsection{香川県ネット・ゲーム依存症対策条例}
これは香川県にて令和2年4月1日(2020年)に施行された未成年のインターネットとコンピュータゲームによる依存症を防止する目的で定められた条例である.
内容としてインターネットやコンピュータゲームの一日の使用時間を一日当たり60分(休業日は90分)に規定し,午後9時(義務教育修了後の子は10時)までにしようをやめさせ,これを家庭や学校で遵守するよう努めなければいけないというものである.

問題となったのは家庭で時間を決めるのではなく条例として規定してしまう点である.
2020年9月に憲法違反だとして提訴されたが,この条例は罰則がないため権利の制約を課すものでないとして棄却された.

ネットやゲーム依存症はWHOによって疾患として認定されそれを元に制定された条例だが,情報社会である現代においてインターネットやコンピュータゲームの利用を制限することはそれらを悪いものとして決めつけている点があると考えられる.
またネット・ゲーム依存症は長時間の利用によるものではなく家庭環境やそれ以外の疾患によるものではないかとされている点から,この条例は疑問視されている.
